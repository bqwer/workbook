\documentclass[a5paper, DIV=14, headings=openany, twoside=true,fontsize=10pt, titlepage]{scrreprt}
\usepackage[T1,T2A]{fontenc}
\usepackage[utf8]{inputenc}
\usepackage[english, russian]{babel} 
\usepackage{indentfirst}
\usepackage{subfig}
\usepackage{graphicx}
\usepackage{float}
\usepackage{import}
\usepackage[rgb]{xcolor}
\usepackage{svg}
\usepackage{listingsutf8}
\usepackage{scrhack}
%\usepackage{pxfonts}
\usepackage{caption}%[2013/01/01]
%\usepackage{inconsolata}
\usepackage{dejavu}
%\lstset{basicstyle=\ttfamily}

%\usepackage{helvet}
%\renewcommand{\familydefault}{\sfdefault}

\definecolor{light-gray}{gray}{0.97}
\renewcommand*\chapterformat{}

\lstset{inputencoding=utf8, extendedchars=\true, basicstyle=\ttfamily}
\lstset{language=Verilog,
        captionpos=b,
        backgroundcolor = \color{light-gray},
        aboveskip=1em,
        belowskip=1em}

\setkomafont{disposition}{\normalfont}
%\addtokomafont{chapter}{\large}
\captionsetup{belowskip=10pt,aboveskip=8pt}
\captionsetup[lstlisting]{justification=centering}

%\pretolerance=2000
%\hyphenpenalty=2000

\usepackage{scrlayer-scrpage}

\let\tempmargin\oddsidemargin
\let\oddsidemargin\evensidemargin
\let\evensidemargin\tempmargin
\reversemarginpar


\newcommand*\xor{\mathbin{\oplus}}
\newcommand{\quotes}[1]{«#1»}

\begin{document}
\rofoot*{\pagemark}
\lofoot*{}
\refoot*{}
\lefoot*{\pagemark}

\sloppy

\title{Лабораторный практикум}
\author{}
\subject{}
\subtitle{\quotes{Проектирование цифровых устройств с помощью Verilog HDL}}
\titlehead{}
\publishers{}
\date{}


\maketitle{}


\chapter{Лабораторная работа №1\\Введение в Verilog HDL} 
\section{Возникновение языков описания цифровой аппаратуры}

\par{Цифровые устройства — это устройства, предназначенные для приёма и обработки цифровых сигналов. Цифровыми называются сигналы, которые можно рассматривать в виде набора дискретных уровней. В цифровых сигналах информация кодируется в виде конкретного уровня напряжения. Как правило выделяется два уровня — логический «0» и логическая «1».}

\par{Цифровые устройства стремительно развиваются с момента изобретения электронной лампы, а затем транзистора. Со временем цифровые устройства стали компактнее, уменьшилось их энергопотребление, возрасла вычислительная мощность. Так же разительно возросла сложность их структуры.}

\par{Графические схемы, которые применялись для проектирования цифровых устройств на ранних этапах развития, уже не могли эффективно использоваться. Потребовался новый инструмент разработки, и таким инструментом стали языки описания аппаратной части цифровых устройств (\foreignlanguage{english}{Hardware Description Languages, HDL}), которые описывали цифровые структуры формализованным языком, чем-то похожим на язык программирования.}

\par{Совершенно новый подход к описанию цифровых схем, реализованный в языках \foreignlanguage{english}{HDL}, заключается в том, что с помощью их помощью можно описывать не только структуру, но и поведение цифрового устройства.Окончательная структура цифрового устройства получается путём обработки таких смешанных описаний специальной программой — синтезатором.}

\par{Такой подход существенно изменил процесс разработки цифровых устройств, превратив громоздкие, тяжело читаемые схемы в относительно простые и доступные описания поведения.}

\par{В данном курсе мы рассмотрим язык описания цифровой аппаратуры \foreignlanguage{english}{Verilog HDL} — одни из наиболее распространённых на текущий момент. И начнём мы с разработки наиболее простых цифровых устройств — логических вентилей.}

\section{\foreignlanguage{english}{HDL} описания логических вентилей}

\par{Логические вентили реализуют функции алгебры логики: И, ИЛИ, Исключающее ИЛИ, НЕ. Напомним их таблицы истинности:}

\begin{table}[!htbp]
\parbox{.45\linewidth}{
\centering
\begin{tabular}{c|c|c}
$a$&$b$&$a \cdot b$\\
\hline
$0$ & $0$ & $0$ \\
$0$ & $1$ & $0$ \\
$1$ & $0$ & $0$ \\
$1$ & $1$ & $1$ \\
\end{tabular}
\caption{И}
} \hfill
\parbox{.45\linewidth}{
\centering
\begin{tabular}{c|c|c}

$a$&$b$&$a | b$\\
\hline
$0$ & $0$ & $0$ \\
$0$ & $1$ & $1$ \\
$1$ & $0$ & $1$ \\
$1$ & $1$ & $1$ \\

\end{tabular}
\caption{ИЛИ}
}\hfill
\parbox{.45\linewidth}{
\centering
\begin{tabular}{c|c|c}
$a$&$b$&$a \xor b$\\
\hline
$0$ & $0$ & $0$ \\
$0$ & $1$ & $1$ \\
$1$ & $0$ & $1$ \\
$1$ & $1$ & $0$ \\
\end{tabular}
\caption{Исключающее ИЛИ}
}\hfill
\parbox{.45\linewidth}{
\centering
\begin{tabular}{c|c}
$a$&$\bar{a}$\\
\hline
$0$ & $1$\\
$1$ & $0$\\
\end{tabular}
\caption{НЕ}
}
\end{table}

\par{Начнём знакомиться с \foreignlanguage{english}{Verilog HDL} с описания логического вентиля \quotes{И}. Ниже приведен код, описывающий вентиль с точки зрения его структуры:}

\newpage
\lstinputlisting[caption={Модуль, описывающий вентиль \quotes{И}}, ]{./code_examples/and_gate.v}

\par{Описанный выше модуль можно представить как некоторый \quotes{ящик}, в который входит 2 провода с названиями \emph{\quotes{\foreignlanguage{english}{a}}} и \emph{\quotes{\foreignlanguage{english}{b}}} и из которого выходит один провод с названием \emph{\quotes{\foreignlanguage{english}{result}}}. Внутри этого блока результат выполнения операции \quotes{И} (в синтаксисе Verilog записывается как \quotes{\&}) над входами соединяют с выходом.}
\par{Схемотично изобразим этот модуль:}

\begin{figure}[H]
\centering
\def\svgwidth{\columnwidth}
\includesvg{pict_1_1}
\caption{Структура модуля \foreignlanguage{english}{\quotes{and\_gate}}}
\end{figure}

\par{Аналогично опишем все оставшиеся вентили:}

\lstinputlisting[caption={Модуль, описывающий вентиль \quotes{ИЛИ}}, ]{./code_examples/or_gate.v}
\lstinputlisting[caption={Модуль, описывающий вентиль \\ \quotes{Исключающее ИЛИ}}, ]{./code_examples/xor_gate.v}
\lstinputlisting[caption={Модуль, описывающий вентиль \quotes{НЕ}}, ]{./code_examples/not_gate.v}

\par{В проектировании цифровых устройств логические вентили наиболее часто используются для формулировки и проверки сложных условий, например:}

\lstinputlisting[caption={Пример использования логических вентилей}, ]{./code_examples/logic_gatex_example.v}

\par{Условие будет выполняться либо когда \emph{не} выполнено условие \emph{\quotes{\foreignlanguage{english}{c}}}, либо когда одновременно выполняются условия \emph{\quotes{\foreignlanguage{english}{a}}} и \emph{\quotes{\foreignlanguage{english}{b}}}. \emph{Здесь и далее под условием понимается логический сигнал, отражающий его истинность.}

\par{В качестве входов, выходов и внутренних соединений в блоках могут использоваться шины — группы проводов. Ниже приведен пример работы с шинами:}

\newpage
\lstinputlisting[caption={Модуль, описывающий побитовое \quotes{ИЛИ} \\ между двумя шинами}, ]{./code_examples/bus_or.v}

\par{Это описание описывает побитовое \quotes{ИЛИ} между двумя шинами по 8 бит. То есть описываются восемь логических вентилей \quotes{ИЛИ}, каждый из которых имеет на входе соответствующие разряды из шины \emph{\quotes{\foreignlanguage{english}{x}}} и шины \emph{\quotes{\foreignlanguage{english}{y}}}.}

\par{При использовании шин можно в описании использовать конкретные биты шины и группы битов. Для этого используют квадратные скобки после имени шины:}

\lstinputlisting[caption={Модуль, демонстрирующий \\ битовую адресацию шин}, ]{./code_examples/bitwise_ops.v}

\newpage
\par{Такому описанию соответствует следующая структурная схема:}

\begin{figure}[H]
\centering
\def\svgwidth{\columnwidth}
\includesvg{pict_1_2}
\caption{Структура модуля \foreignlanguage{english}{\quotes{bitwise\_ops}}}
\end{figure}

\chapter{Лабораторная работа №6\\FLASH память} 
\section{Возникновение языков описания цифровой аппаратуры}

\end{document}