\documentclass[a5paper, DIV=14, headings=openany, twoside=true,fontsize=10pt, titlepage]{scrreprt}
\usepackage[T1,T2A]{fontenc}
\usepackage[utf8]{inputenc}
\usepackage[english, russian]{babel} 
\usepackage{indentfirst}
\usepackage{subfig}
\usepackage{graphicx}
\usepackage{float}
\usepackage{import}
\usepackage[rgb]{xcolor}
\usepackage{svg}
\usepackage{listingsutf8}
\usepackage{scrhack}
%\usepackage{float}
%\usepackage{pxfonts}
\usepackage{caption}%[2013/01/01]
%\usepackage{inconsolata}
\usepackage{dejavu}
\usepackage[justification=centering]{caption}
%\lstset{basicstyle=\ttfamily}
\usepackage{varwidth}
\definecolor{light-gray}{gray}{0.97}
\renewcommand*\chapterformat{}
\renewcommand\textbullet{\ensuremath{\bullet}}

\usepackage{tikz-timing}
%\usetikztiminglibrary[rising arrows]{clockarrows}
\usetikztiminglibrary[arrow tip=latex, rising arrows]{clockarrows}
\usetikztiminglibrary{nicetabs}
\usepackage{xparse} % NewDocumentCommand, IfValueTF, IFBooleanTF

\lstset{inputencoding=utf8, extendedchars=\true, basicstyle=\ttfamily}
\lstset{language=Verilog,
        captionpos=b,
        backgroundcolor = \color{light-gray},
        aboveskip=1em,
        belowskip=1em}

\setkomafont{disposition}{\normalfont}
%\addtokomafont{chapter}{\large}
\captionsetup{belowskip=10pt,aboveskip=8pt}
%\captionsetup[lstlisting]{justification=centering}
%\lstloadlanguages{Verilog}
%\pretolerance=2000
%\hyphenpenalty=2000

\usepackage{scrlayer-scrpage}

\let\tempmargin\oddsidemargin
\let\oddsidemargin\evensidemargin
\let\evensidemargin\tempmargin
\reversemarginpar

\newcommand{\nsig}[1]{$\overline{\mbox{#1}}$}
\newcommand*\xor{\mathbin{\oplus}}
\newcommand{\quotes}[1]{«#1»}
\newcommand{\eng}[1]{\foreignlanguage{english}{#1}}
\newcommand{\qeng}[1]{\quotes{\foreignlanguage{english}{#1}}}

\begin{document}
\rofoot*{\pagemark}
\lofoot*{}
\refoot*{}
\lefoot*{\pagemark}

\sloppy

\title{Лабораторный практикум}
\author{}
\subject{}
\subtitle{\quotes{Проектирование цифровых устройств с помощью \eng{Verilog HDL}}}
\titlehead{}
\publishers{}
\date{}


\maketitle{}


\chapter{Лабораторная работа №1\\Введение в Verilog HDL} 
\section{Возникновение языков описания цифровой аппаратуры}

\par{Цифровые устройства — это устройства, предназначенные для приёма и обработки цифровых сигналов. Цифровыми называются сигналы, которые можно рассматривать в виде набора дискретных уровней. В цифровых сигналах информация кодируется в виде конкретного уровня напряжения. Как правило выделяется два уровня — логический «0» и логическая «1».}

\par{Цифровые устройства стремительно развиваются с момента изобретения электронной лампы, а затем транзистора. Со временем цифровые устройства стали компактнее, уменьшилось их энергопотребление, возрасла вычислительная мощность. Так же разительно возросла сложность их структуры.}

\par{Графические схемы, которые применялись для проектирования цифровых устройств на ранних этапах развития, уже не могли эффективно использоваться. Потребовался новый инструмент разработки, и таким инструментом стали языки описания аппаратной части цифровых устройств (\eng{Hardware Description Languages, HDL}), которые описывали цифровые структуры формализованным языком, чем-то похожим на язык программирования.}

\par{Совершенно новый подход к описанию цифровых схем, реализованный в языках \eng{HDL}, заключается в том, что с помощью их помощью можно описывать не только структуру, но и поведение цифрового устройства.Окончательная структура цифрового устройства получается путём обработки таких смешанных описаний специальной программой — синтезатором.}

\par{Такой подход существенно изменил процесс разработки цифровых устройств, превратив громоздкие, тяжело читаемые схемы в относительно простые и доступные описания поведения.}

\par{В данном курсе мы рассмотрим язык описания цифровой аппаратуры \eng{Verilog HDL} — одни из наиболее распространённых на текущий момент. И начнём мы с разработки наиболее простых цифровых устройств — логических вентилей.}

\section{\eng{HDL} описания логических вентилей}

\par{Логические вентили реализуют функции алгебры логики: И, ИЛИ, Исключающее ИЛИ, НЕ. Напомним их таблицы истинности:}

\begin{table}[!htbp]
\parbox{.45\linewidth}{
\centering
\begin{tabular}{c|c|c}
$a$&$b$&$a \cdot b$\\
\hline
$0$ & $0$ & $0$ \\
$0$ & $1$ & $0$ \\
$1$ & $0$ & $0$ \\
$1$ & $1$ & $1$ \\
\end{tabular}
\caption{И}
} \hfill
\parbox{.45\linewidth}{
\centering
\begin{tabular}{c|c|c}

$a$&$b$&$a | b$\\
\hline
$0$ & $0$ & $0$ \\
$0$ & $1$ & $1$ \\
$1$ & $0$ & $1$ \\
$1$ & $1$ & $1$ \\

\end{tabular}
\caption{ИЛИ}
}\hfill
\parbox{.45\linewidth}{
\centering
\begin{tabular}{c|c|c}
$a$&$b$&$a \xor b$\\
\hline
$0$ & $0$ & $0$ \\
$0$ & $1$ & $1$ \\
$1$ & $0$ & $1$ \\
$1$ & $1$ & $0$ \\
\end{tabular}
\caption{Исключающее ИЛИ}
}\hfill
\parbox{.45\linewidth}{
\centering
\begin{tabular}{c|c}
$a$&$\bar{a}$\\
\hline
$0$ & $1$\\
$1$ & $0$\\
\end{tabular}
\caption{НЕ}
}
\end{table}

\par{Начнём знакомиться с \eng{Verilog HDL} с описания логического вентиля \quotes{И}. Ниже приведен код, описывающий вентиль с точки зрения его структуры:}

\begin{minipage}{\linewidth}
\lstinputlisting[caption={Модуль, описывающий вентиль \quotes{И}}]{./code_examples/and_gate.v}
\end{minipage}

\par{Описанный выше модуль можно представить как некоторый \quotes{ящик}, в который входит 2 провода с названиями \emph{\qeng{a}} и \emph{\qeng{b}} и из которого выходит один провод с названием \emph{\qeng{result}}. Внутри этого блока результат выполнения операции \quotes{И} (в синтаксисе Verilog записывается как \quotes{\&}) над входами соединяют с выходом.}
\par{Схемотично изобразим этот модуль:}

\begin{figure}[H]
\centering
\def\svgwidth{\columnwidth}
\includesvg{pict_1_1}
\caption{Структура модуля \qeng{and\_gate}}
\end{figure}

%\begin{minipage}{\linewidth}
\par{Аналогично опишем все оставшиеся вентили:}

\begin{minipage}{\linewidth}
\lstinputlisting[caption={Модуль, описывающий вентиль \quotes{ИЛИ}}]{./code_examples/or_gate.v}
\end{minipage}
\begin{minipage}{\linewidth}
\lstinputlisting[caption={Модуль, описывающий вентиль \mbox{\quotes{Исключающее ИЛИ}}}]{./code_examples/xor_gate.v}
\end{minipage}
\begin{minipage}{\linewidth}
\lstinputlisting[caption={Модуль, описывающий вентиль \quotes{НЕ}}]{./code_examples/not_gate.v}
\end{minipage}

\par{В проектировании цифровых устройств логические вентили наиболее часто используются для формулировки и проверки сложных условий, например:}

\begin{minipage}{\linewidth}
\lstinputlisting[caption={Пример использования логических вентилей}, ]{./code_examples/logic_gatex_example.v}
\end{minipage}

\par{Условие будет выполняться либо когда \emph{не} выполнено условие \emph{\qeng{c}}, либо когда одновременно выполняются условия \emph{\qeng{a}} и \emph{\qeng{b}}. \emph{Здесь и далее под условием понимается логический сигнал, отражающий его истинность.}}
\par{В качестве входов, выходов и внутренних соединений в блоках могут использоваться шины — группы проводов. Ниже приведен пример работы с шинами:}

\begin{minipage}{\linewidth}
\lstinputlisting[caption={Модуль, описывающий побитовое \quotes{ИЛИ} \mbox{между двумя шинами}}, ]{./code_examples/bus_or.v}
\end{minipage}

\par{Это описание описывает побитовое \quotes{ИЛИ} между двумя шинами по 8 бит. То есть описываются восемь логических вентилей \quotes{ИЛИ}, каждый из которых имеет на входе соответствующие разряды из шины \emph{\qeng{x}} и шины \emph{\qeng{y}}.}

\par{При использовании шин можно в описании использовать конкретные биты шины и группы битов. Для этого используют квадратные скобки после имени шины:}

\begin{minipage}{\linewidth}
\lstinputlisting[caption={Модуль, демонстрирующий \mbox{битовую адресацию шин}}, ]{./code_examples/bitwise_ops.v}
\end{minipage}

%\begin{samepage}
\begin{minipage}{\textwidth}
\par{Такому описанию соответствует следующая структурная схема:}

\begin{figure}[H]
\centering
\def\svgwidth{\columnwidth}
\includesvg{pict_1_2}
\caption{Структура модуля \qeng{bitwise\_ops}}
\end{figure}

%\end{samepage}
\end{minipage}

\begin{figure}[H]
\begin{tikztimingtable}[%
    timing/dslope=0.1,
    timing/.style={x=5ex,y=2ex},
    x=5ex,
    timing/rowdist=3ex,
    timing/name/.style={font=\sffamily\scriptsize}
]
CLK         & 18{c} \\
FRAME     & 2u 6L H U \\
\nsig{AD} [31:0]   & 2u 1D{addr} 1U{} 1D{$d_1$} D{$d_1 '$} D{$d_2$} 2D{$d_3$} U \\
C/BE[3:0] & 2u 1D{0010} 6D{BE\#} U  \\
IRDY      & UU 4L HLH \\
TRDY       & UU HLH 3L H \\
DEVSEL     & 2U 6L H\\
\extracode
\begin{pgfonlayer}{background}
\begin{scope}[semitransparent ,semithick]
\vertlines[darkgray,dotted]{0.5,1.5 ,...,8.0}
\end{scope}
\end{pgfonlayer}
\end{tikztimingtable}
\caption{Временная диаграмма операции чтения шины \eng{PCI}}
\end{figure}

\chapter{Лабораторная работа №6\\\eng{FLASH} память} 
\section{Возникновение \eng{FLASH}-памяти}
\par{Ни один из блоков цифровых устройств, которые мы рассмотрели ранее не способен хранить информацию при отсутствии питания.}
\par{На заре вычислительной техники, данные в цифровое устройство после подачи питания загружали с таких носителей, как перфокарты и, позже, магнитные ленты. Ещё позже для целей хранения информации при отсутствии питания были разработаны накопители на жёстких магнитных дисках, которые известны нам по аббревиатуре \qeng{HDD}.}
\par{Энергонезависимые накопители информации обладают как преимуществами, так и недостатками по сравнению с энергозависимой памятью.} 
\par{Как правило, энергонезависимая память существенно уступает по скорости работы \eng{RAM}-памяти. Это ограничение удалось преодолеть только недавно: в 2016 году широкой общественности была представлена постоянная память, где информация хранится в виде спина электрона. Такая память по скорости работы не уступает современной \eng{RAM}-памяти, такой как \eng{DDR5}. Но подобная память ещё долгое время будет оставаться недоступной для рядового пользователя из-за высокой стоимости.}
\par{На данный момент наиболее широко используемая энергонезависимая память - это память типа \eng{FLASH}.}
\par{В качестве элемента хранения информации такая память использует транзистор с плавающим затвором, где состояние затвора определяет бит хранимой информации.}
%\includesvg{pict_6_1}
\begin{figure}[H]
\centering
\def\svgwidth{\columnwidth}
\includesvg{no_image}
\caption{Ячейка \eng{FLASH}-памяти}
\end{figure}


\begin{samepage}
\par{Из-за использования транзистора с плавающим затвором, у \eng{FLASH}-памяти есть характерные особенности:
  \begin{itemize}
    \item Запись значения возможна только из логической \quotes{1} в логический \quotes{0};
    \item Удаление информации возможно из группы ячеек одновременно (блока);
    \item Удаление и запись информации приводят к деградации ячеек памяти;
    \item Чтение также приводит к деградации ячеек памяти, но в меньшей степени.
  \end{itemize}}
\end{samepage}

\par{Из-за физических процессов протекающих в транзисторах во время записи значений в память и очистке содержимого.}
\end{document}
